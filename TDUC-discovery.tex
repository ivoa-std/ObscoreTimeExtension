
%%  Discovery of data products for Time domain use cases
\lstset{captionpos=t}
\begin{itemize}
\item  Finding a light curve in a time interval for a sky position 
 \begin{lstlisting} [language=SQL, captionpos=t, caption=Show me a list of all data matching a particular event (gamma ray burst) in time interval and space ]
    I. DataType='light-curve'
   II. RA includes 16.00 hours
  III. DEC includes +41.00
   IV. Time start > MJD 55220 and Time stop < MJD 55221
    V. Number of time slots > 1000 
  \end{lstlisting}

\item  Times series for a sky position, with date, length and exposure constraints
\begin{lstlisting} [language=SQL, caption=Show me a list of all data which satisfies]
    I. DataType='time-series'
   II. RA includes 16.00 hours
  III. DEC includes +41.00
  IV. Time resolution better than 1 minute
   V. Time interval (start of series to end of series) > 1 week
  VI. Observation data before June 10, 2008
 VII. Observation data after June 10, 2007      
  \end{lstlisting}   
    
\item  Finding a light curve in folded mode for pulsar analysis 
\begin{lstlisting} [language=SQL, caption=Show me a list of all data matching a light curve for a pulsar candidate]
    I. DataType='light-curve'
   II. time resolution <  0.001 s 
   III. time axis is folded 
   IV. exposure time > 5s
   \end{lstlisting}  
   
\item  Finding MUSE cube time series
\begin{lstlisting} [language=SQL, caption=Show me a list of all data products from MUSE data collection with more than 30 items]
    I. DataType='time-cube'
   II. Data collection like  'MUSE'
  III. Number of time slots > 30
  \end{lstlisting}
 
 %looking for Pulsar data  
\begin{lstlisting} [language=SQL, caption= Show me a list of all data matching a light curve for a radio source ]
    I. DataType='light-curve'
   II. Band corresponds to Radio frequency min> 30MHz and frequency max < 30 GHz
  %em_min > radio_min   and em_max < radiomax 
   III. Minimum time sample exposure > 3s 
   IV. Number of time slots > 10
   \end{lstlisting} 
   
 % trouver des light_curve comparables � celles de ma liste de source qui sont en TDB Barycenter
 \begin{lstlisting} [language=SQL, caption=Show me a list of all data products using a specified Time system ]
   Show me a list of all data products using a specified Time system  
    I. DataType='light-curve' or 'time-series'
    II. time scale=TDB
    III. time reference position=BARYCENTER 
  \end{lstlisting}
 
 % identifier des transits de planetes 
 TBC planet transit ? TESS ?? 
 
 % identifier des systemes d'�toiles binaires 
TBC binary stars ??
 
 % nature article https://doi.org/10.1038/s41586-023-06787-x
 \item  Here is an example of the data discovery steps one would launch in the VO for looking at specific binary systems 
 in the supernova SN 2022jli \citep{2024Natur.625..253C}
 
 % A 12.4-day periodicity in a close binary system after a supernova
 % target position in ICRS 00 34 45.690 -08 23 12.16 %
 % object name = SN 2022jli
 \begin{lstlisting} [language=SQL, caption=Show me a list of light curves around object   \emph{SN 2022jli}]
    I. DataType='light-curve' 
    II. target position close to SN 2022jli
    III. em_min >  10  and em_max <  1.0E-8 % radio  and Xray , gammaray
    VI. Observation data before Sept 31, 2023
    VII. Observation data after Sept 01, 2022    
   \end{lstlisting}
   
  Check what the Fermi-Lat telescope may have seen in the mean time 
  \begin{lstlisting} [language=SQL, caption=Show me a list of light curves around object   \emph{SN 2022jli}]
     I. DataType='light-curve' 
    II. Data collection like Fermi-Lat 
    IV. t_min > 59823     %Observation data before sept 31, 2023
    V. t_max < 60218   % Observation data after sept 01, 2022
  \end{lstlisting}
  
   \begin{lstlisting} [language=SQL, caption=Show me a list of dynamic spectra around object   \emph{SN 2022jli}]]
     I. DataType='dynamic-spectrum'
    II. target position close to SN 2022jli
    \end{lstlisting} 
  
  \end{itemize}